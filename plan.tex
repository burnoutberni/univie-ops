%% LyX 2.1.3 created this file.  For more info, see http://www.lyx.org/.
%% Do not edit unless you really know what you are doing.
\documentclass[english]{article}
\usepackage[T1]{fontenc}
\usepackage[latin9]{inputenc}
\usepackage{geometry}
\geometry{verbose,tmargin=4cm,bmargin=4cm,lmargin=4cm,rmargin=4cm}

\makeatletter
%%%%%%%%%%%%%%%%%%%%%%%%%%%%%% User specified LaTeX commands.
\usepackage{blindtext}

\makeatother

\usepackage{babel}
\begin{document}

\title{OPS\\
Nichtlineare Optimierung ohne Nebenbedingungen (Mehrdimensionale Verfahren):
Planung}


\author{Team 4\\
Hayden Bernhard\\
Freynschlag Felix\\
Pressler Christoph\\
Biedermann Sonja\\
Kharin Stepan}

\maketitle

\section*{Arbeitsteilung (vorl�ufig)}


\paragraph{Theorieausarbeitung}
\begin{enumerate}
\item Nelder-Mead-Verfahren: X Y
\item Abstiegsverfahren: X Z
\item Gradientenverfahren: Christoph Pressler
\item Koordinatenabstiegsmethode: Felix Freynschlag
\end{enumerate}

\paragraph{Implementierung}
\begin{enumerate}
\item Nelder-Mead-Verfahren: Sonja Biedermann
\item Abstiegsverfahren: Sonja Biedermann
\item Gradientenverfahren: Christoph Pressler
\end{enumerate}

\paragraph{Homepage}
\begin{enumerate}
\item Grafiken: Sonja Biedermann
\item Gestaltung der Webseite: Bernhard Hayden, Christoph Pressler
\item Text: Sonja Biedermann
\item Beispiele: X X
\end{enumerate}

\section*{Nelder-Mead-Verfahren}

\blindtext


\subsection*{M�gliche Quellen}
\begin{itemize}
\item \texttt{https://de.wikipedia.org/wiki/Downhill-Simplex-Verfahren}
\item \texttt{http://www.math.uni-hamburg.de/home/oberle/skripte/}~\\
\texttt{optimierung/optim04.pdf}
\item \texttt{http://hgs.iwr.uni-heidelberg.de/expdesign/teaching/ss12/}~\\
\texttt{optimization/Handout\%20-\%20Die\%20Simplexmethode.pdf}
\item Vorlesungsskript
\end{itemize}

\section*{Abstiegsverfahren}

\blindtext


\subsection*{M�gliche Quellen}
\begin{itemize}
\item \texttt{http://www.math.uni-hamburg.de/home/oberle/skripte/}~\\
\texttt{optimierung/optim05.pdf}
\item \texttt{http://www2.math.uni-paderborn.de/fileadmin/Mathematik/}~\\
\texttt{People/walther/seminar/Vortrag\_Gradientenverfahren.pdf}
\item \texttt{https://www.tu-ilmenau.de/fileadmin/media/num/neundorf/}~\\
\texttt{Dokumente/Preprints/abstieg1.pdf}
\item Vorlesungsskript
\end{itemize}

\section*{Gradientenverfahren}

Grunds�tzlich macht sich das Gradientenverfahren den Fakt zunutze,
dass der Gradient immer in die Richtung des st�rksten Anstiegs zeigt.
Dementsprechend zeigt nat�rlich der negative Gradient in die Richtung
des st�rksten Abstiegs. Nun \quotedblbase geht\textquotedblleft{}
man eine gewisse Zeit in diese Richtung bis man einen neuen Punkt
hat von dem man abermals den Gradienten ausrechnet. Die unterschiedlichen
Gradientenverfahren unterscheiden sich nun darin wie weit man in die
Richtung des negativen Gradienten gehen soll. Im theoretischen Teil
wird sowohl st�rker auf das Gradientenverfahren an sich, als auch
auf die Unterschiede zwischen den verschiedenen Arten eingegangen
werden.


\subsection*{M�gliche Quellen}
\begin{itemize}
\item \texttt{https://de.wikipedia.org/wiki/Gradientenverfahren}
\item \texttt{http://www.math.uni-hamburg.de/home/oberle/skripte/}~\\
\texttt{optimierung/optim06.pdf}
\item \texttt{http://www.iue.tuwien.ac.at/phd/bauer/node63.html}
\item Vorlesungsskript
\end{itemize}

\section*{Koordinatenabstiegsmethode}

Bei der Koordinatenabstiegsmethode wird in jedem Schritt eine Koordinate
$i$ gew�hlt, nach der optimiert wird. Der Startpunkt ist beliebig
w�hlbar und der folgende Vorgang wird mehrmals wiederholt. Bei jedem
Durchgang werden alle Koordinaten des aktuellen Punktes $x^{k}$ in
die Funktion eingesetzt, nur die gew�hlte Koordinate $i$ beh�lt man
als Variable bei. Die dadurch entstandene eindimensionale Funktion
wird minimiert und die Koordinate \emph{$i$ }wird mit dem gefundenen
Wert ersetzt. Dieser Vorgang wird mit allen Koordinaten wiederholt.
Nun kann man diesen gesamten Vorgang erneut anwenden, bis sich der
Punkt nicht mehr oder nur kaum ver�ndert; danach wird abgebrochen.


\subsection*{M�gliche Quellen}
\begin{itemize}
\item \texttt{http://www.qucosa.de/fileadmin/data/qucosa/documents/}~\\
\texttt{11484/Projektarbeit\_Andr\%C3\%A9\_Clausner.pdf}
\item Numerische Verfahren zur Nichtlinearen Optimierung aus Algorithmische
Mathematik \textendash{} Winfried Hochst�tter
\item Vorlesungsskript\end{itemize}

\end{document}
\grid
